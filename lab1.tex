%% Introduction to latex facilities.
%% Sat 31 Dec 2005
%% Stephen Eglen.

%% Text following a percent sign (%) until the end of line is treated
%% as a comment.

\documentclass{article}

%%%%%%%%%%%%%%%%%%%%%%%%%%%%%%%%%%%%%%%%%%%%%%%%%%%%%%%%%%%%%%%%%%%%%%
%% This section is called the preamble, where we can specify which
%% latex packages we required.  Most (but not of all) of the packages
%% below should be fairly standard in most latex documents.  The
%% exception is xspace and the new \latex command, which you probably
%% do not need.
%%%%%%%%%%%%%%%%%%%%%%%%%%%%%%%%%%%%%%%%%%%%%%%%%%%%%%%%%%%%%%%%%%%%%%

%% Better math support:
\usepackage{amsmath}

%% Bibliography style:
\usepackage{mathptmx}           % Use the Times font.
\usepackage{graphicx}           % Needed for including graphics.
\usepackage{url}                % Facility for activating URLs.

%% Set the paper size to be A4, with a 2cm margin 
%% all around the page.
\usepackage[a4paper,margin=2cm]{geometry}

%% Natbib is a popular style for formatting references.
\usepackage{natbib}
%% bibpunct sets the punctuation used for formatting citations.
\bibpunct{(}{)}{;}{a}{,}{,}

%% textcomp provides extra control sequences for accessing text symbols:
\usepackage{textcomp}
\newcommand*{\micro}{\textmu}
%% Here, we define the \micro command to print a text "mu".
%% "\newcommand" returns an error if "\micro" is already defined.

%% This is an example of a new macro that I've created to save me
%% having to type \LaTeX each time.  The xspace command provides space
%% after the word LaTeX where appropriate.
\usepackage{xspace}
\providecommand*{\latex}{\LaTeX\xspace}
%% "\providecommand" does nothing if "\latex" is already defined.


%%%%%%%%%%%%%%%%%%%%%%%%%%%%%%%%%%%%%%%%%%%%%%%%%%%%%%%%%%%%%%%%%%%%%%
%% Start of the document.
%%%%%%%%%%%%%%%%%%%%%%%%%%%%%%%%%%%%%%%%%%%%%%%%%%%%%%%%%%%%%%%%%%%%%%

\begin{document}
\author{authors\\
  Stuff\\
  For Fun}
\date{\today}
\title{LAB 1}
\maketitle


\section{Stuff}

\section{Other Stuff}

\section{And more Stuff}

\subsection{default}
Total Cycles:                           2498189 (4.99638 msec)\\
Execution Cycles:                         2122494 ( 84.96\%)\\
Stall Cycles:                              375695 ( 15.04\%)\\
Nops:                                      153215 (  6.13\%)\\
Executed operations:                      2656192\\

\textbf{Parameters}\\
\# 2 issue vex 1 cluster \\
CFG: Debug 0\\
RES: IssueWidth 2\\
RES: Alu.0 2 \\
RES: Memory.0 1 \\ 
RES: Mpy.0 2  \\
RES: CopySrc.0 2 \\ 
RES: CopyDst.0 2 \\

\subsection{VeryWide}

Total Cycles:                             2147918 (4.29584 msec)\\
Execution Cycles:                         1769735 ( 82.39\%)\\
Stall Cycles:                              378183 ( 17.61\%)\\
Nops:                                      265686 ( 12.37\%)\\
Executed operations:                      2860932\\

\textbf{Parameters}\\

CFG: Debug \\
RES: IssueWidth 32\\
RES: MemLoad 32\\
RES: MemStore 32\\
RES: MemPft 32\\

RES: Alu.0 32 \\
RES: Memory.0 32\\ 
RES: Mpy.0 32 \\
RES: CopySrc.0 32\\ 
RES: CopyDst.0 32\\
x 4 clusters




\end{document}